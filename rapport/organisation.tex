%ORGANISATION
\section{Cahier des charges}
    \subsection{Environnement de travail}
        Pour réaliser ce projet nous avons choisi le langage \textbf{C++} pour sa proximité avec le C et l'apport de l'orienté objet.
        Nous avons utilisé les bibliothèques standards du C++ ainsi
        que la bibliothèque graphique \textbf{Allegro} en version 5.0.
        Les différents membres de l'équipe ont codé sur environnement \textbf{Linux} ou \textbf{OSX}, les logiciels et technologies suivantes ont été utilisées :
        
        \begin{itemize}
            \item \textbf{Édition du code} : Atom 
            \item \textbf{Synchronisation} : Github  
            \item \textbf{Communication} : Slack, Skype
            \item \textbf{UML} : Dia
            \item \textbf{Rapport de projet} : ShareLaTeX 
        \end{itemize}

    \subsection{Organisation}
        L'équipe que nous formons pour mener à bien ce projet est constituée de :
        
        \begin{itemize}
            \item Bastien LEPESANT et Thomas LEFEBVRE pour la partie apprentissage par réseaux de neurones.
            \item Filipe GAMA, Lucas NICOSIA et Vincent MONOT pour la partie génération procédurale de créatures via algorithme génétique.

        \end{itemize}
        
    \subsection{Objectifs}
        \subsubsection{Fonctionnalités envisagées}
        \subsubsection{Limites}
    \subsection{Répartition des tâches}
    \subsection{Livrables}
        Les rapport devra être livré avant le 7 mars au soir, les autres livrables seront délivrés après la soutenance ayant lieu le 10 mars à 11h25 :
        
        \begin{itemize}
            \item Rapport de projet commun aux 2 parties du projet
            \item Code documenté et à jour de chaque partie du projet
            \item Exécutable pour les 2 parties du projet
            \item Slides utilisées lors de la soutenance
        \end{itemize}
