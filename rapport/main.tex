\documentclass[a4paper, 12pt]{report}

\usepackage[utf8]{inputenc}
\usepackage[T1]{fontenc}
\usepackage{babel}
%\usepackage[latin]{babel}
\usepackage{bookmark,hyperref}
\usepackage{glossaries}

\newglossaryentry{reseaudeneurone}
{
    name=reseaudeneurone,
    description={fdsfsqfqs}
}

\usepackage{natbib}
\bibliographystyle{abbrvnat}

\renewcommand{\bibname}{Bibliographie}
\renewcommand{\contentsname}{Sommaire}
\renewcommand{\partname}{Partie}

\usepackage{titling}

\newcommand{\subtitle}[1]{%
  \posttitle{%
    \par\end{center}
    \begin{center}\large#1\end{center}
    \vskip0.5em}%
}

\pretitle{%
  \begin{center}
  \LARGE
  \includegraphics[width=10cm,height=5cm]{ucp.jpeg}\\[\bigskipamount]
}

\title{Rapport final}
\subtitle{Génération de créatures procédurale et apprentissage par réseaux de neurones}
\author{LEFEBVRE Thomas et LEPESANT Bastien}
\date{10 Mars 2016}

\usepackage{natbib}
\usepackage{graphicx}

\makeglossaries

\begin{document}

\maketitle
\tableofcontents

%INTRODUCTION
\section{Introduction}

Ce projet de synthèse a été réalisé dans le cadre de la L3 Informatique à l'université de Cergy-Pontoise.\\\\
le sujet "Génération de créatures procédurale et apprentissage par réseaux de neurones" a été proposé et choisi par notre groupe car réunissant les intérêts commun de chacun d'entre nous et des sujets qu'il nous paraissait probant d'aborder et d'étudier (en grande partie par nous même) en cette 3ème année à l'université.\\\\
L'utilisation d'algorithmes génétiques en ce qui concerne la génération des créatures et le mécanisme d'apprentissage mais également l'apport des réseaux de neurones sont les bases de notre projet et nous expliquerons plus en avant leur fonctionnement et la mise en pratique de ces concepts dans ce rapport.



%ORGANISATION
\section{Cahier des charges}
    \subsection{Environnement de travail}
        Pour réaliser ce projet nous avons choisi le langage \textbf{C++} pour sa proximité avec le C et l'apport de l'orienté objet.
        Nous avons utilisé les bibliothèques standards du C++ ainsi
        que la bibliothèque graphique \textbf{Allegro} en version 5.0.
        Les différents membres de l'équipe ont codé sur environnement \textbf{Linux} ou \textbf{OSX}, les logiciels et technologies suivantes ont été utilisées :
        
        \begin{itemize}
            \item \textbf{Édition du code} : Atom 
            \item \textbf{Synchronisation} : Github  
            \item \textbf{Communication} : Slack, Skype
            \item \textbf{UML} : Dia
            \item \textbf{Rapport de projet} : ShareLaTeX 
        \end{itemize}

    \subsection{Organisation}
        L'équipe que nous formons pour mener à bien ce projet est constituée de :
        
        \begin{itemize}
            \item Bastien LEPESANT et Thomas LEFEBVRE pour la partie apprentissage par réseaux de neurones.
            \item Filipe GAMA, Lucas NICOSIA et Vincent MONOT pour la partie génération procédurale de créatures via algorithme génétique.

        \end{itemize}
        
    \subsection{Objectifs}
        \subsubsection{Fonctionnalités envisagées}
        \subsubsection{Limites}
    \subsection{Répartition des tâches}
    \subsection{Livrables}
        Les rapport devra être livré avant le 7 mars au soir, les autres livrables seront délivrés après la soutenance ayant lieu le 10 mars à 11h25 :
        
        \begin{itemize}
            \item Rapport de projet commun aux 2 parties du projet
            \item Code documenté et à jour de chaque partie du projet
            \item Exécutable pour les 2 parties du projet
            \item Slides utilisées lors de la soutenance
        \end{itemize}

\part{Apprentissage par réseaux de neurones}
%PARTIE THEORIQUE DE L'APPRENTISSAGE PAR RESEAUX DE NEURONES
\section{Théorie}
%PARTIE PRATIQUE DE L'APPRENTISSAGE PAR RESEAUX DE NEURONES
\section{Pratique}
\part{Génération des créatures}
%PARTIE THEORIQUE DE LA CREATION DE CREATURE PROCEDURALE
\section{Théorie}
%PARTIE PRATIQUE DE LA GENERATION DE CREATURE PROCEDURALE
\section{Pratique}

\bookmarksetup{startatroot}
\addtocontents{toc}{\bigskip}

\include{conclusion}
%REMERCIEMENTS (ça parait important)
\section{Remerciements}
Nous aimerions tout d'abord remercier les enseignants encadrants lors de ce projet, Philippe LAROQUE ainsi que Tuyêt Trâm DANG NGOC mais également Tianxiao LIU pour sa participation au jury de la soutenance.\\\\
Nous souhaiterions également remercier Pierre ANDRY pour le cours sur les réseaux de neurones qui nous a permis d'avoir une première approche sur ce sujet pour le moins complexe.\\\\
Enfin nous tenons à exprimer notre gratitude envers l'ensemble de la promotion L3 Informatique 2015/2016 pour l'intérêt porté à ce projet et le potentiel soutien apporté.

\nocite{*}
\bibliography{references}

\part{Annexes}

\printglossaries
%\include{glossaire}
%HISTOIRE (en gros l'origine des algos génétiques et des réseaux de neurones, ça peut être intéressant à rajouter en annexes)
\section{Un peu d'hisoire}




\end{document}
